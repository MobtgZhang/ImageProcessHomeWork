% !TeX program = xelatex
% !TeX encoding = UTF-8
\documentclass{HighSchoolBigDataCompetition}
\usepackage{zhlipsum,mwe}
\cansaibianhao{bcd210722}
\groupclassifier{研究生组}
\timu{大数据挑战赛 \LaTeX{} 模板写作示例}
\keyword{content;content;content}
\begin{document}
	\begin{abstract}
		请使用 \TeX Live 2019,XeLaTeX 编译,请选用支持 UTF­-8 编码的编辑器。

		使用者需要有一定的 \LaTeX{} 的使用经验({\heiti 任务期三个月以内不建议使用 \LaTeX},因此本文没有介绍基础使用),至少要会使用常用宏包的一些功能,比如参考文献,数学公式,图片使用,列表环境等等。模板已经添加了常用的宏包,无需用户再额外添加。
		
		本模板
		\begin{itemize}
			\item 定义了几个宏 \lstinline|\def\ee{\mathrm{e}},\def\ii{\mathrm{i}},\def\leq{\leqslant},\def\geq{\geqslant}| 方便使用;
			\item 图片应放在 \lstinline|figure| 文件夹中;
			\item 定制了 matlab 和 python 代码环境,使用方法:\lstinline|\begin{matlab} content \end{matlab}| 和 \lstinline|\begin{python} content \end{python}|;
			\item 加载了 \lstinline|cleveref| 宏包,使用方法:\lstinline|\cref{label}|。
		\end{itemize}
		其它的就是跟普通的 \lstinline|ctexart| 使用方法一样。
	\end{abstract}
	\tableofcontents\newpage
	\section{问题提出}
	\subsection{问题背景}
	异常检测(异常诊断/发现)、异常预测、趋势预测,是智能运维中首当其冲需要解决的问题。这类问题是通过业务、系统、产品直接关联的 KPI 业务指标进行分析诊断,指标主要包括用户感知类(如页面打开延时)、服务性能(如用户点击量)、服务器硬件健康状况(如 CPU 利用率、内存使用率)等关键性能指标。

	不同场景的运维,分析的指标种类差异较大,但都具备时序性特点,不同场景的 KPI 指标,以毫秒、秒、分钟、小时、天为时间间隔的数据序列都会出现,有些复杂场景的业务,往往会混合多个时间间隔的数据,但均为随时间变化而变化的时序数据。
	\subsection{问题重述}
	本文中以运营商基站KPI的性能指标为研究数据,数据是从数据是从2021年8月28日0时至9月25日23时共29天5个基站覆盖的58个小区对应的67个KPI指标。其中,作为数据的核心属性,这里选择三个核心的指标进行分析:
	\begin{enumerate}
		\item 第一个指标指的是消去内的平均用户数量,表示的是某基站覆盖消去一定时间内通过收集在线的平均用户人数;
		\item 第二个指标指的是小区 PDCP 流量,通过小区 PDCP 层所发送的下行数据的总吞吐量(比特)与 小区 PDCP 层所接收到的上行数据的总吞吐量(比特)两个指标求和得到,表示某基站覆盖的小区在一定时间内的上下行流量总和;
		\item 第三个指标:平均激活用户数,表示某基站覆盖的小区在一定时间内曾经注册过无线网络的平均人数。
	\end{enumerate}
	通过以上的三个指标,我们来回答以下的三个问题:
	\begin{enumerate}
		\item \textbf{问题1,异常检测}:这里利用到的是附件中的指标数据,对所有消去在上述三个关键指标上检测处这29天内共有对多少个异常数值,其中异常数值包含有以下的两种情况:异常孤立点与异常周期,然后汇总所有的消去异常情况填写表格信息.
		\item \textbf{问题2,异常预测}:针对问题1针对问题 1 检测出的异常数值,通过该异常数值前的数据建立预测模型,预测未来是否会发生异常数值.
		\item \textbf{问题3,趋势预测}:利用2021年8月28日0时至9月25日23时已有的数据,预测未来三天(即9月26日0时-9月28日23时)上述三个指标的取值。
	\end{enumerate}
	\section{问题分析}

	\subsection{问题整体分析}

	\subsection{问题一的分析}

	\subsection{问题二的分析}

	\subsection{问题三的分析}

	\section{模型假设}
	\begin{enumerate}
		\item content;
		\item content;
		\item content;
		\item content;
		\item content
	\end{enumerate}
	\section{符号说明}
	\begin{center}
		\begin{tabularx}{0.7\textwidth}{c@{\hspace{1pc}}|@{\hspace{2pc}}X}
			\Xhline{0.08em}
			符号 & \multicolumn{1}{c}{符号说明}\\
			\Xhline{0.05em}
			$\delta$ & 赤纬角\\
			$\beta$ & 经度\\
			$\alpha$ & 纬度\\
			$r$ & 地球半径\\
			$\gamma$ & 太阳光与杆所成的夹角\\
			$l$ & 杆的长度\\
			$l_{y}$ & 杆的影子长度\\
			$\vec{x}_{1},\vec{y}_{1},\vec{z}_{1}$ & 由杆的位置所生成的切平面的正交基\\
			$\vec{\hat{x}}_{1},\vec{\hat{y}}_{1},\vec{\hat{z}}_{1}$ & 由杆的位置所生成的切平面的单位正交基\\
			$\theta$ & 影子与北方的夹角\\
			$l_{y}(i)$ & 编号为 $i$ 的数据对应的影子长度\\
			$\theta_{i}$ & 编号为 $i$ 的数据对应的影子角度\\
			\Xhline{0.08em}
		\end{tabularx}
	\end{center}

	\section{模型的建立与求解}
	\subsection{数据分析与预处理}
	\subsection{问题一解决方案}
	效果见\cref{fig:1}。
	\begin{figure}[htbp]
		\centering
		\includegraphics{example-image-plain.pdf}
		\caption{content}\label{fig:1}
	\end{figure}
	\subsection{问题二解决方案}
	结果见\cref{tab:1}。
	\begin{table}[htbp]
		\centering
		\caption{content}\label{tab:1}
		\begin{tabularx}{0.7\textwidth}{c@{\hspace{1pc}}|@{\hspace{2pc}}X}
		\Xhline{0.08em}
		符号 & \multicolumn{1}{c}{符号说明}\\
		\Xhline{0.05em}
		$\delta$ & 赤纬角\\
		$\beta$ & 经度\\
		$\alpha$ & 纬度\\
		$r$ & 地球半径\\
		$\gamma$ & 太阳光与杆所成的夹角\\
		$l$ & 杆的长度\\
		$l_{y}$ & 杆的影子长度\\
		$\vec{x}_{1},\vec{y}_{1},\vec{z}_{1}$ & 由杆的位置所生成的切平面的正交基\\
		$\vec{\hat{x}}_{1},\vec{\hat{y}}_{1},\vec{\hat{z}}_{1}$ & 由杆的位置所生成的切平面的单位正交基\\
		$\theta$ & 影子与北方的夹角\\
		$l_{y}(i)$ & 编号为 $i$ 的数据对应的影子长度\\
		$\theta_{i}$ & 编号为 $i$ 的数据对应的影子角度\\			\Xhline{0.08em}
		\end{tabularx}
	\end{table}
	\subsection{问题三解决方案}
	

	\section{模型评价与推广}
	\subsection{模型的优点}
	
	\subsection{模型的缺点}
	
	\subsection{模型的改进}
	\cite{label}

	\phantomsection
	\addcontentsline{toc}{section}{参考文献}
	\begin{thebibliography}{99}
	\bibitem{label}content
	\end{thebibliography}

	\newpage
	\appendix
	\ctexset{section={
		format={\zihao{-4}\heiti\raggedright}
	}}
	\begin{center}
		\heiti\zihao{4} 附\hspace{1pc}录
	\end{center}
	\section{问题一的 MATLAB 代码}
	\begin{matlab}
clc,clear
%第七题
R71 = 1;
R72 = 2;
T7 = 1;
K7 = 1;
N7=10^5;
G71=tf(R72,R71);
G72=tf(1,[T7 1]);
G73=tf(1,[T7 0]);
G74=tf([N7*T7 0],[T7 N7]);
G75=tf([N7*K7*T7 N7*K7],[T7 N7]);
G76=tf([K7*T7 K7],[T7 0]);

subplot(2,3,1)
step(G71)
xlabel('$t$','interpreter','latex', 'FontSize', 12);
ylabel('$y$','interpreter','latex', 'FontSize', 12);
title('比例环节');

subplot(2,3,2)
step(G72)
xlabel('$t$','interpreter','latex', 'FontSize', 12);
ylabel('$y$','interpreter','latex', 'FontSize', 12);
title('惯性环节');

subplot(2,3,3)
step(G73)
xlabel('$t$','interpreter','latex', 'FontSize', 12);
ylabel('$y$','interpreter','latex', 'FontSize', 12);
title('积分环节');

subplot(2,3,4)
step(G74,10^-3)
xlabel('$t$','interpreter','latex', 'FontSize', 12);
ylabel('$y$','interpreter','latex', 'FontSize', 12);
title('微分环节');

subplot(2,3,5)
step(G75,10^-3)
xlabel('$t$','interpreter','latex', 'FontSize', 12);
ylabel('$y$','interpreter','latex', 'FontSize', 12);
title('比例微分环节');

subplot(2,3,6)
step(G76)
xlabel('$t$','interpreter','latex', 'FontSize', 12);
ylabel('$y$','interpreter','latex', 'FontSize', 12);
title('比例积分环节');
	\end{matlab}
	\section{问题一的 Julia 代码}
	\begin{julia}
		function myfunction(a,b;c=50)

		end
	\end{julia}
\end{document}